\documentclass[
  a4paper,            % DIN A4
  DIV=10,             % Schriftgröße und Satzspiegel
  oneside,            % einseitiger Druck
  BCOR=5mm,           % Bindungskorrektur
  parskip=half,       % Halber Abstand zwischen Absätzen
  numbers=noenddot,   % Kein Punkt hinter Kapitelnummern
  bibtotoc,           % Literaturverzeichnis im Inhaltsverzeichnis
  listof=totoc        % Abbildungs- und Tabellenverzeichnis im Inhaltsverzeichnis
]{scrreprt}
\usepackage{style/thesisstyle}
\usepackage{graphicx}
\usepackage{tabularx} % Das Paket bietet Tools zum Gestalten schönerer Tabellen an. 
\usepackage{glossaries}
\usepackage{listings} % Paket für Source-Code Abschnitte
\usepackage{tabularx} % Das Paket bietet Tools zum Gestalten schönerer Tabellen an. 

\usepackage{float}
\usepackage{enumitem}

\usepackage[column=0]{cellspace}
\setlength{\cellspacetoplimit}{\tabcolsep}
\setlength{\cellspacebottomlimit}{\cellspacetoplimit}
\addparagraphcolumntypes{X}

\newlist{tabenum}{enumerate}{1}
\setlist[tabenum]{label*=\arabic*.,
                  labelwidth=2em, 
                  leftmargin=2em, 
                  nosep,
                  before=\begin{minipage}[t]{\hsize},
                  after=\end{minipage}}

% template %
\newcommand\addrow[2]{#1 & #2\\ \hline}

\newcommand\additemizedrow[2]{#1 &
        \begin{tabenum}
            #2
        \end{tabenum}
        \\ \hline}

% making stuff convenient %
\newcommand\name[1]{\addrow{Name}{#1}}
\newcommand\actor[1]{\addrow{Actor}{#1}}
\newcommand\udescription[1]{\addrow{Description}{#1}}
\newcommand\precondition[1]{\addrow{Precondition}{#1}}
\newcommand\scenario[1]{\additemizedrow{Scenario}{#1}}
\newcommand\result[1]{\addrow{Result}{#1}}
\newcommand\extensions[1]{\additemizedrow{Extensions}{#1}}
\newcommand\exceptions[1]{\additemizedrow{Exceptions}{#1}}

\newenvironment{usecase}{\tabularx{\textwidth}{|0{wl{3cm}}|0{X}|}\hline}{\endtabularx}
\setlength{\parindent}{0em}
\setlength{\parskip}{1em}

\newcommand{\mulcom}[1]{} % Verwenden, um mehrere Zeilen auszukommentieren
\renewcommand{\arraystretch}{2.5} 

\makeglossaries           % create all glossary entries (remember: run makeglossaries manually)
\loadglsentries{thesisglossaries.tex}  % load acronym, symbol and glossary entries

\sisetup{locale = DE}     % siunitx locale setup
%\DeclareSIUnit \fps{fps}  % a custom unit (usage: \SI{24}{\fps})

\begin{document}

% !TEX root = ../thesis.tex
%
% configurations
%

% English Language support
% -> uncomment if needed
% Beta!
%\fullenglish{yes}
\fullenglish{no}

% text field
%-> replace supervisor names with correct ones
\firstSupervisor{Prof. Dr. Thomas Clemen}
\secondSupervisor{}

% text field
%-> replace title with your thesis title
\thesisTitle{Entwicklung eines digitalen Zwillings einer Tello im MAS}
\thesisTitleEN{Implementation of a digital twin framework for unmanned aerial vehicles}

% text field
%-> replace the key words with your own key words
\keywordsDE{Digitaler Zwilling, Human-Drone Interaction, Quadrocopter, Unbemanntes Luftfahrzeug, Jogging}
\keywordsEN{Digital Twin, Human-Drone Interaction, Quadrocopter, Unmanned aerial vehicle, Jogging}

% text field
%-> replace the text with a description of the thesis
\abstractDE{PLATZHALTER}
\abstractEN{PLATZHALTER}

% text field
%-> replace john with your name
\thesisAuthor{Leon Chun Wai Yuen}

% text field
%-> enter the submission date
\submissionDate{PLATZHALTER}

% switch - uncomment only one
%-> uncomment NDA or public
%\NDA{yes}
\NDA{no}

% switch - uncomment only one
%-> uncomment old standard cover or cover Corporate Design 2017
\Cover{CD2017}
%\Cover{CD2017NoLogo}
%\Cover{Std2018}
%\Cover{Std2018_green} 			% with green bar

% switch - uncomment only one
%-> uncomment to show list of figures or not
\ListOfFigures{yes}
%\ListOfFigures{no}

% switch - uncomment only one
%-> uncomment to show list of tables or not
\ListOfTables{yes}
%\ListOfTables{no}

% switch - uncomment only one
%-> uncomment to show list of accronyms or not
\ListOfAccronyms{yes}
%\ListOfAccronyms{no}

% switch - uncomment only one
%-> uncomment to show list of symbols or not
\ListOfSymbols{yes}
%\ListOfSymbols{no}

% switch - uncomment only one
%-> uncomment to show list of glossary entries or not
\Glossary{yes}
%\Glossary{no}

% switch - uncomment only one
%-> uncomment the study course you are in
\studycourse{ITS}
%\studycourse{TI}
%\studycourse{AI}
%\studycourse{WI}
%\studycourse{EI}
%\studycourse{REE}
%\studycourse{BMP}		
%\studycourse{BMP-hp}	 % Internship Report in M&P
%\studycourse{BMT}
%\studycourse{BMT-st}    % Study / home assignment in BMT
%\studycourse{BMT-hp}    % Internship Report in BMT
%\studycourse{MI}
%\studycourse{MIK}
%\studycourse{MA}
    % load all settings

\hyphenation{Ba-che-lor-the-sis Mas-ter-the-sis}

% Cover page here, no page numberj
\ICoverPage

% PDF Metadata
% !TEX root = ../thesis.tex
%
% PDF Metadata integration
% @author Thomas Lehmann
%

% PDF Metadata
\hypersetup{
pdftitle={\IthesisTitle},
pdfauthor={\IthesisAuthor},
pdfkeywords={\IkeyWordsEN}
}

% Titlepage is page one even if the number is not shown.
\pagenumbering{roman}
% Title page here
% !TEX root = ../thesis.tex
%
% title page
% @author Thomas Lehmann
% Hints for title page and page numbering: https://en.wikipedia.org/wiki/Title_page
%
\title{\IthesisTitle}   % set latex default title to be used by hyperref in pdf
\author{\IthesisAuthor} % set latex default author to be used by hyperref in pdf

\newpage
\thispagestyle{empty}
{\fontfamily{phv}\selectfont
  \hfuzz=20pt       % suppress warnings due to extension onto page margins

  % Author of thesis
  \vspace*{1cm}
  \begin{minipage}[b]{\textwidth}
    \fontsize{14pt}{20pt}
    \selectfont
    \begin{center}
      \IthesisAuthor
    \end{center}
  \end{minipage}

  % Title of thesis
  \vspace{1.5cm}
  \begin{minipage}[b][0cm][t]{\textwidth}
    \fontsize{18pt}{20pt}
    \selectfont
    \begin{center}
      \IthesisTitle
    \end{center}
  \end{minipage}

  % Important information
  \begin{textblock*}{\textwidth}(40mm,210mm)
    \begin{minipage}[b]{\textwidth}
      \hbadness=10001    % suppress underfull warning due to short text
      \fontfamily{cmr}\selectfont
      \fontsize{12pt}{14pt}
      \selectfont
      \ifdefined\ILanguageEN
        \IthesisKindEN ~submitted for examination in \IthesisExaminationEN \\
        in the study course \textit{\IstudyCourseName} \\
        at the \IthesisDepartmentFullEN \\
        at the \IthesisFacultyFullEN \\
        at University of Applied Science Hamburg\\

        Supervisor: \IfirstSv \\
        \ifdefined\IisTermPaper
          % left blank
        \else
          \ifdefined\IisInternshipReport
	  Supervised: \IsecondSv\\
          \else
        Supervisor: \IsecondSv \\
          \fi\fi
        
        Submitted on: \ISubDate \\
      \else
      	\ifdefined\IisInternshipReport
        \IthesisKindDE ~eingereicht im Rahmen des \IthesisExaminationDE \\	
	\else
        \IthesisKindDE ~eingereicht im Rahmen der \IthesisExaminationDE \\
        \fi
	im Studiengang \textit{\IstudyCourseName} \\
        am \IthesisDepartmentFull \\
        der \IthesisFacultyFull \\
        der Hochschule für Angewandte Wissenschaften Hamburg\\

        Betreuender Prüfer: \IfirstSv \\
        \ifdefined\IisTermPaper
          % left blank
        \else
          \ifdefined\IisInternshipReport
        betriebliche Betreuung: \IsecondSv \\							
	  \else
        Zweitgutachter: \IsecondSv \\
        \fi\fi

        Eingereicht am: \ISubDate \\
      \fi
    \end{minipage}
  \end{textblock*}
}


% Abstract page here
% !TEX root = ../thesis.tex
%
% abstract page
% @author Leon Yuen
%
\newpage
\thispagestyle{plain}
\clearpage
\hfuzz=12pt       % suppress warnings due to extenstion onto page margins

\ifdefined\ILanguageEN
  % just skip
\else
    \textbf{\IthesisAuthor}

    \vspace{0.3cm}
    \textbf{Thema der Arbeit}

    \IthesisTitle

    \vspace{0.3cm}
    \textbf{Stichworte}

    \IkeyWordsDE

    \vspace{0.3cm}
    \textbf{Kurzzusammenfassung}

    \begin{minipage}{\textwidth}
    \IabstractDE
    \end{minipage}
\fi

\vspace{1.0cm}
\textbf{\IthesisAuthor}

\vspace{0.3cm}
\textbf{Title of Thesis}

\IthesisTitleEN

\vspace{0.3cm}
\textbf{Keywords}

\begin{minipage}{\textwidth}
\IkeyWordsEN
\end{minipage}

\vspace{0.3cm}
\textbf{Abstract}

\IabstractEN


\clearpage

\section*{Danksagung}

% Table of contents here
\tableofcontents

% List of figures here
\IListOfFigures

% List of tables here
\IListOfTables

% List of accronyms here
\IListOfAccronyms

% List of symbols here
\IListOfSymbols

% Uncomment if list of source code is needed (rarely).
%\lstlistoflistings  % requires package listings, needs to uncommenting of usepackage

% path to the chapters folder is set to find the images used there
\graphicspath{ {./chapters/} }

% Chapters
\clearpage
\pagenumbering{arabic}
\chapter{Einleitung}

\section{Motivation}

Anfänglich nur für militärische Einsatzzwecke entwickelt, werden Unmanned Aerial Vehicle, zu deutsch Unbemanntes Luftfahrzeug heute für verschiedene Anwendungsgebiete umfunktioniert.  Die Human-Drohne Interaction untersucht die Interaktion zwischen UAV und Laien und haben in der Vergangenheit vorgeschlagen UAV als Social Companions einzusetzen \cite{Tezza2019StateOfTheArtHumanDroneInter}. Vorhergegangene Arbeiten haben ein positives Bild von UAV in der Öffentlichkeit gezeigt (Paper zitieren) und ein Interesse als Begleiter bei sportlichen Aktivtäten. Diese Arbeit setzt auf die Vorarbeit von 
\section{Ziel der Arbeit}

Für die Bachelorarbeit soll ein Digitaler Zwillings für ein Unbemanntes Luftfahrzeug (UAV) im MARS Framework konzeptioniert und entwickelt werden. Der Digitale Zwilling soll das Verhalten des UAV in einer digitale Simulation abbilden und ermöglichen das Fahrzeug durch weitere Funktionalitäten zu erweitern. Dabei soll Architektur des Digitalen Zwillings nicht domänenspezifisch sein. \newline
Die Entwicklung lässt sich in drei Aufgabenteile unterteilen.
Im ersten Teil wird ein Datenmodell des UAV entwickelt. Hierfür werden Sensordaten des physikalischen UAV gesammelt und im Datenmodell gespeichert. Diese Informationen werden ausgewertet und einem Zustand zugeordent. \newline
Der zweite Teil der Aufgabe besteht aus der Entwicklung einer digitalen Entität, die geometrisch und physikalischen den realen UAV darstellt. Hierbei wird eine Visualisierung erstellt, die das Verhalten des UAV animiert. \newline
Im letzten Teil soll ein Anwendungsbeispiel entwickelt werden. Das Beispiel soll das fertige System demonstrieren und dient der Systemevaluation.

Für die Bachelorarbeit soll ein Digitaler Zwilling entwickelt werden, der einen Menschen beim Joggen begleiten soll und seinen Ausdauerzustand modelliert.Konzeptioniert und Entwickelt im MARS Framework, soll der Digitale Zwilling  Der Digitale Zwilling analysiert über einen Fitness Tracker und über die Distanz einer 


%Die Arbeit lässt sich in drei Teilbereiche untergliedern.

% Die erste Teilaufgabe besteht aus dem Sammeln von Daten und der Erstellung eines Repräsentativen Zustandsgraphen. Dazu muss der Digitale Zwilling kontinuierlich die Zustandsinformationen des Quadrocopter abfragen und sie in das eigene Datenmodell abbilden.
% Ggf müssen aus den Informationen weiter Informationen berechnet werden. Mit hoher Wahrscheinlichkeit können die meisten Daten nur über die Kamerabilder ermittelt werden, weil die Drohne sonst kaum Informationen liefert

% Der dritten Teil wird die aquirierten Daten aus dem Datenmodell ausgewertet und der Zustand des physikalischen Objekts bestimmt. um Repräsentation der Zustände des physikalischen Objekts

% \input{chapters/0introduction/begriffe.tex}
\section{Abgrenzung}

% Möchte keine simulationsfähigen Digitalen Zwilling erstellen 
\section{Gliederung der Arbeit}


\chapter{Begriffserklärung}


Dieses Kapitel stellt in relevanten Grundlagen der Bachelorarbeit vor und dienen dem verständnis der nachfolgenden Kapitel. 

\section{Digitaler Zwilling}

    Das Konzept des Digitalen Zwilling wurde 2002 von Micheal Grieves eingeführt und beschreibt die Modellierung eines physischen, nicht domainspezifischen Objekts in einenm virtuellen Raum.  

    Die virtuelle Repräsentation des physischen Objekts ermöglicht die Simulation unterschiedlicher Szenarien, um die bestmögliche Konfiguration des Systems für den gegebenen Zustand vorzunehmen und sie auf das physischen Objekt anzuwenden. 

    \textbf{physischer Raum} \\

    \textbf{Virtuelle Raum} \\

    \textbf{Modell des Digitalen Zwillings} \\

    \textbf{physischer Umgebung} \\

    \textbf{Virtuelle Umgebung} \\

    \subsection{Definition und Geschichte}

    \subsection{Anwendungen}

    \subsection{Digitaler Zwilling für Unbemannte Luftfahrzeuge}

\section{Multi-Agent System}

\section{MARS Framgework}

    \subsection{Multi-Agend System with Mars}

    \subsubsection{Agent}

    \subsubsection{layer}

% welche technologischen Vorteile erhoffte sich Grieves?





\input{chapters/MaterialUndMethoden/MaterialUndMethoden.tex}
\chapter{Anforderungsanalyse}

\section{Funktionale Anforderungen}

\renewcommand{\arraystretch}{1.4}
\begin{table}[H]
\begin{tabular}{ p{0.5cm} p{3cm} p{11cm}}
    ID & Komponente & Beschreibung  \\
    \hline
    F01 & Digitaler Zwilling & Der DZ fragt in periodischen Abständen Statuswerte des physischen UAV. \\
    F02 & Digitaler Zwilling & Der DZ speichert alle erfassten Statuswerte des physischen UAV ab. \\
    F03 & Digitaler Zwilling & Der DZ seinen letzten Zustand wiederherstellen, wenn der Zustand mit dem des PZ übereinstimmt. \\
    F04 & Digitaler Zwilling & Endet oder bricht die Kommunikation zwischen dem PZ und dem DZ ab, wird der letzte Zustand des UAV gespeichert und der UAV landet zum automatischen Landen befohlen. \\
    F05 & Digitaler Zwilling & Fall der physischen UAV seinen Zustand ändert, passt sich der Zustand des DZ an. \\
    F06 & Phys. UAV & Falls der DZ seinen Zustand aktualisiert, wird die Änderung an den physischen UAV gesendet und ausgeführt. \\
    F07 & Phys. UAV & Falls die Batterie der PZ bei 10Prozent liegt, wird der UAV automatisch gelandet. \\
    F08 & Visualisierung & Das virtuelle UAV besitzt die gleichen kinetischen und physischen Eigenschaften wie die des physischen UAV. \\
    F09 & Visualisierung & Alle Bewegungen und Zustandsänderungen des physischen UAV werden vom virtuellen UAV gespiegelt. \\
    F10 & Visualisierung & Für die Visualisierung des physischen UAV im virtuellen Raum wird ein virtueller UAV erstellt. \\
    F11 & Kommunikation & Falls der DZ gestartet wird, verbindet sich der DZ mit dem physischen UAV. \\
    F12 & Kommunikation & Falls der DZ sich nach 5 Sekunden keine Verbindung zum PZ aufbauen kann, wird der Prozess beendet. \\
    F13 & Benutzerausgabe & Die Videoübertragung des physischen UAV wird in Echtzeit übertragen. \\
    F14 & Benutzerausgabe & Der Zustand und die Attributel des physischen UAV werden auf dem Rechner angezeigt.\\
    F15 & Fehlerzustand & Falls das physische UAV nach 5 gesenden Befehlen keine Antwortnachricht zuruckliefert, wird der physische zum Landen befohlen. \\
    F16 & Fehlerzustand & Falls der DZ einen unbekannten Fehlercode erhält, wird das physikalsche UAV zum Landen befohlen. \\
    F17 & Fehlerzustand & Falls sich der physische UAV in einem unbekannten Zustand befindet, wird sie zum Landen befohlen.
\end{tabular}
\caption{Funktionale Anforderungen}\label{table:Funktionale Anforderungen}
\end{table}

\subsection{Use Case Beschreibung}

\begin{usecase}
    \name{Digitalen Zwilling und UAV verbinden}
    \actor{UAV, Benutzer, System}
    \udescription{Starten und verbinden der Drohne mit dem  Digitalen Zwilling}
    \precondition{Das UAV ist eingeschaltet. Der Computer ist über WLan mit dem UAV verbunden}
    \scenario{
        \item System show something
        \item User do this
        \item System do that
    }
    \result{Digitaler Zwilling und UAV sind verbunden.}
    \extensions{
        \item[3a] If something do something
    }
    \exceptions{
        \item[2.1] System message: "Nope"
        \item[2.2] System message: "Bad action"
    }
\end{usecase}

\mulcom{
    \begin{usecase}
        \name{Titel}
        \actor{UAV, Benutzer, System}
        \udescription{Starten und verbinden der Drohne mit dem  Digitalen Zwilling}
        \precondition{Das UAV ist eingeschaltet. Der Computer ist über WLan mit dem UAV verbunden}
        \scenario{
            \item System show something
            \item User do this
            \item System do that
        }
        \result{Digitaler Zwilling und UAV sind verbunden.}
        \extensions{
            \item[3a] If something do something
        }
        \exceptions{
            \item[2.1] System message: "Nope"
            \item[2.2] System message: "Bad action"
        }
    \end{usecase}
}

\newpage

\section{Nicht funktionale Anforderungen}

\mulcom {
\begin{table}[h]
    \begin{tabular}{ l p{12cm}}
        ID & Begriff  \\
        \hline
        NF01 & Der Digitale Zwilling verarbeitet nur aktuelle Nachrichten und verwirft alte, die außerhalb des validen Zeitraums liegen. \\
        NF02 & Der Digitale Zwilling wählt optimale Operationen zum gegebenen Wissen aus. \\ 
        NF03 & Das System kann durch verschiedene Funktionalitäten erweitert werden. \\      
    \end{tabular}
    \caption{Nicht funktionale Anforderungen}\label{table:nicht Funktionale Anforderungen}
    
\end{table}
}

\textbf{Erweiterbarkeit}

\textbf{Zuverlässig}


% \section{Use Case}

% Test Szenarien

\chapter{Konzeption}

\section{Systemarchitektur}

\subsection{Quadroter}
% Was sind die Tools die ich Nutze?
% .Net
% Mars
% Drohne
% Testing tool
% Modellierung

\subsection{Datenabfrage}
\subsection{Datenmodell}
\chapter{Implementierung}
\chapter{Evaluation}

\section{Codeevaluation}

Das System wird durch Unit Tests auf Korrektheit geprüft. Das Testen soll eine frühzeitige Erkennung von Bugs gewährleisten und das System auf Korrektheit verifizieren. Für das überprüfen der Testszenarien wird das NUnit Framework verwendet. 


\chapter{Diskussion und Ausblick}

\section{Zusammenfassung}
\section{Diskussion}

\section{Ausblick}



% Add additional chapters here

\bibliographystyle{plain}
%\bibliographystyle{dinat}
\bibliography{literature}

% Appendix
\appendix
% !TEX root = ../thesis.tex
% appendix example chapter
% @author Thomas Lehmann
%
\chapter{Anhang}


\IGlossary

\Istatement

\end{document}
