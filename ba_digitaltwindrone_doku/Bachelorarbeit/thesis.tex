\documentclass[
  a4paper,            % DIN A4
  DIV=10,             % Schriftgröße und Satzspiegel
  oneside,            % einseitiger Druck
  BCOR=5mm,           % Bindungskorrektur
  parskip=half,       % Halber Abstand zwischen Absätzen
  numbers=noenddot,   % Kein Punkt hinter Kapitelnummern
  bibtotoc,           % Literaturverzeichnis im Inhaltsverzeichnis
  listof=totoc        % Abbildungs- und Tabellenverzeichnis im Inhaltsverzeichnis
]{scrreprt}
\usepackage{style/thesisstyle}
\usepackage{graphicx}
\usepackage{tabularx} % Das Paket bietet Tools zum Gestalten schönerer Tabellen an. 
\usepackage{glossaries}
\usepackage{listings} % Paket für Source-Code Abschnitte
\usepackage{tabularx} % Das Paket bietet Tools zum Gestalten schönerer Tabellen an. 

\usepackage{float}
\usepackage{enumitem}

\usepackage[column=0]{cellspace}
\setlength{\cellspacetoplimit}{\tabcolsep}
\setlength{\cellspacebottomlimit}{\cellspacetoplimit}
\addparagraphcolumntypes{X}

\newlist{tabenum}{enumerate}{1}
\setlist[tabenum]{label*=\arabic*.,
                  labelwidth=2em, 
                  leftmargin=2em, 
                  nosep,
                  before=\begin{minipage}[t]{\hsize},
                  after=\end{minipage}}

% template %
\newcommand\addrow[2]{#1 & #2\\ \hline}

\newcommand\additemizedrow[2]{#1 &
        \begin{tabenum}
            #2
        \end{tabenum}
        \\ \hline}

% making stuff convenient %
\newcommand\name[1]{\addrow{Name}{#1}}
\newcommand\actor[1]{\addrow{Actor}{#1}}
\newcommand\udescription[1]{\addrow{Description}{#1}}
\newcommand\precondition[1]{\addrow{Precondition}{#1}}
\newcommand\scenario[1]{\additemizedrow{Scenario}{#1}}
\newcommand\result[1]{\addrow{Result}{#1}}
\newcommand\extensions[1]{\additemizedrow{Extensions}{#1}}
\newcommand\exceptions[1]{\additemizedrow{Exceptions}{#1}}

\newenvironment{usecase}{\tabularx{\textwidth}{|0{wl{3cm}}|0{X}|}\hline}{\endtabularx}
\setlength{\parindent}{0em}
\setlength{\parskip}{1em}

\newcommand{\mulcom}[1]{} % Verwenden, um mehrere Zeilen auszukommentieren
\renewcommand{\arraystretch}{2.5} 

\makeglossaries           % create all glossary entries (remember: run makeglossaries manually)
\loadglsentries{thesisglossaries.tex}  % load acronym, symbol and glossary entries

\sisetup{locale = DE}     % siunitx locale setup
%\DeclareSIUnit \fps{fps}  % a custom unit (usage: \SI{24}{\fps})

\begin{document}

% !TEX root = ../thesis.tex
%
% configurations
%

% English Language support
% -> uncomment if needed
% Beta!
%\fullenglish{yes}
\fullenglish{no}

% text field
%-> replace supervisor names with correct ones
\firstSupervisor{Prof. Dr. Thomas Clemen}
\secondSupervisor{}

% text field
%-> replace title with your thesis title
\thesisTitle{Entwicklung eines digitalen Zwillings einer Tello im MAS}
\thesisTitleEN{Implementation of a digital twin framework for unmanned aerial vehicles}

% text field
%-> replace the key words with your own key words
\keywordsDE{Digitaler Zwilling, Human-Drone Interaction, Quadrocopter, Unbemanntes Luftfahrzeug, Jogging}
\keywordsEN{Digital Twin, Human-Drone Interaction, Quadrocopter, Unmanned aerial vehicle, Jogging}

% text field
%-> replace the text with a description of the thesis
\abstractDE{PLATZHALTER}
\abstractEN{PLATZHALTER}

% text field
%-> replace john with your name
\thesisAuthor{Leon Chun Wai Yuen}

% text field
%-> enter the submission date
\submissionDate{PLATZHALTER}

% switch - uncomment only one
%-> uncomment NDA or public
%\NDA{yes}
\NDA{no}

% switch - uncomment only one
%-> uncomment old standard cover or cover Corporate Design 2017
\Cover{CD2017}
%\Cover{CD2017NoLogo}
%\Cover{Std2018}
%\Cover{Std2018_green} 			% with green bar

% switch - uncomment only one
%-> uncomment to show list of figures or not
\ListOfFigures{yes}
%\ListOfFigures{no}

% switch - uncomment only one
%-> uncomment to show list of tables or not
\ListOfTables{yes}
%\ListOfTables{no}

% switch - uncomment only one
%-> uncomment to show list of accronyms or not
\ListOfAccronyms{yes}
%\ListOfAccronyms{no}

% switch - uncomment only one
%-> uncomment to show list of symbols or not
\ListOfSymbols{yes}
%\ListOfSymbols{no}

% switch - uncomment only one
%-> uncomment to show list of glossary entries or not
\Glossary{yes}
%\Glossary{no}

% switch - uncomment only one
%-> uncomment the study course you are in
\studycourse{ITS}
%\studycourse{TI}
%\studycourse{AI}
%\studycourse{WI}
%\studycourse{EI}
%\studycourse{REE}
%\studycourse{BMP}		
%\studycourse{BMP-hp}	 % Internship Report in M&P
%\studycourse{BMT}
%\studycourse{BMT-st}    % Study / home assignment in BMT
%\studycourse{BMT-hp}    % Internship Report in BMT
%\studycourse{MI}
%\studycourse{MIK}
%\studycourse{MA}
    % load all settings

\hyphenation{Ba-che-lor-the-sis Mas-ter-the-sis}

% Cover page here, no page numberj
\ICoverPage

% PDF Metadata
% !TEX root = ../thesis.tex
%
% PDF Metadata integration
% @author Thomas Lehmann
%

% PDF Metadata
\hypersetup{
pdftitle={\IthesisTitle},
pdfauthor={\IthesisAuthor},
pdfkeywords={\IkeyWordsEN}
}

% Titlepage is page one even if the number is not shown.
\pagenumbering{roman}
% Title page here
% !TEX root = ../thesis.tex
%
% title page
% @author Thomas Lehmann
% Hints for title page and page numbering: https://en.wikipedia.org/wiki/Title_page
%
\title{\IthesisTitle}   % set latex default title to be used by hyperref in pdf
\author{\IthesisAuthor} % set latex default author to be used by hyperref in pdf

\newpage
\thispagestyle{empty}
{\fontfamily{phv}\selectfont
  \hfuzz=20pt       % suppress warnings due to extension onto page margins

  % Author of thesis
  \vspace*{1cm}
  \begin{minipage}[b]{\textwidth}
    \fontsize{14pt}{20pt}
    \selectfont
    \begin{center}
      \IthesisAuthor
    \end{center}
  \end{minipage}

  % Title of thesis
  \vspace{1.5cm}
  \begin{minipage}[b][0cm][t]{\textwidth}
    \fontsize{18pt}{20pt}
    \selectfont
    \begin{center}
      \IthesisTitle
    \end{center}
  \end{minipage}

  % Important information
  \begin{textblock*}{\textwidth}(40mm,210mm)
    \begin{minipage}[b]{\textwidth}
      \hbadness=10001    % suppress underfull warning due to short text
      \fontfamily{cmr}\selectfont
      \fontsize{12pt}{14pt}
      \selectfont
      \ifdefined\ILanguageEN
        \IthesisKindEN ~submitted for examination in \IthesisExaminationEN \\
        in the study course \textit{\IstudyCourseName} \\
        at the \IthesisDepartmentFullEN \\
        at the \IthesisFacultyFullEN \\
        at University of Applied Science Hamburg\\

        Supervisor: \IfirstSv \\
        \ifdefined\IisTermPaper
          % left blank
        \else
          \ifdefined\IisInternshipReport
	  Supervised: \IsecondSv\\
          \else
        Supervisor: \IsecondSv \\
          \fi\fi
        
        Submitted on: \ISubDate \\
      \else
      	\ifdefined\IisInternshipReport
        \IthesisKindDE ~eingereicht im Rahmen des \IthesisExaminationDE \\	
	\else
        \IthesisKindDE ~eingereicht im Rahmen der \IthesisExaminationDE \\
        \fi
	im Studiengang \textit{\IstudyCourseName} \\
        am \IthesisDepartmentFull \\
        der \IthesisFacultyFull \\
        der Hochschule für Angewandte Wissenschaften Hamburg\\

        Betreuender Prüfer: \IfirstSv \\
        \ifdefined\IisTermPaper
          % left blank
        \else
          \ifdefined\IisInternshipReport
        betriebliche Betreuung: \IsecondSv \\							
	  \else
        Zweitgutachter: \IsecondSv \\
        \fi\fi

        Eingereicht am: \ISubDate \\
      \fi
    \end{minipage}
  \end{textblock*}
}


% Abstract page here
% !TEX root = ../thesis.tex
%
% abstract page
% @author Thomas Lehmann
%
\newpage
\thispagestyle{plain}
\clearpage
\hfuzz=12pt       % suppress warnings due to extenstion onto page margins

\ifdefined\ILanguageEN
  % just skip
\else
    \textbf{\IthesisAuthor}

    \vspace{0.3cm}
    \textbf{Thema der Arbeit}

    \IthesisTitle

    \vspace{0.3cm}
    \textbf{Stichworte}

    \IkeyWordsDE

    \vspace{0.3cm}
    \textbf{Kurzzusammenfassung}

    \begin{minipage}{\textwidth}
    \IabstractDE
    \end{minipage}
\fi

\vspace{1.0cm}
\textbf{\IthesisAuthor}

\vspace{0.3cm}
\textbf{Title of Thesis}

\IthesisTitleEN

\vspace{0.3cm}
\textbf{Keywords}

\begin{minipage}{\textwidth}
\IkeyWordsEN
\end{minipage}

\vspace{0.3cm}
\textbf{Abstract}

\IabstractEN


\clearpage

\input{chapters/danksagung.tex}

% Table of contents here
\tableofcontents

% List of figures here
\IListOfFigures

% List of tables here
\IListOfTables

% List of accronyms here
\IListOfAccronyms

% List of symbols here
\IListOfSymbols

% Uncomment if list of source code is needed (rarely).
%\lstlistoflistings  % requires package listings, needs to uncommenting of usepackage

% path to the chapters folder is set to find the images used there
\graphicspath{ {./chapters/} }

% Chapters
\clearpage
\pagenumbering{arabic}
\chapter{Einleitung}

\section{Motivation}

Anfänglich nur für militärische Einsatzzwecke entwickelt, werden Unmanned Aerial Vehicle, zu deutsch Unbemanntes Luftfahrzeug heute für verschiedene Anwendungsgebiete umfunktioniert.  Die Human-Drohne Interaction untersucht die Interaktion zwischen UAV und Laien und haben in der Vergangenheit vorgeschlagen UAV als Social Companions einzusetzen \cite{Tezza2019StateOfTheArtHumanDroneInter}. Vorhergegangene Arbeiten haben ein positives Bild von UAV in der Öffentlichkeit gezeigt (Paper zitieren) und ein Interesse als Begleiter bei sportlichen Aktivtäten. Diese Arbeit setzt auf die Vorarbeit von 
\section{Zielsetzung}

Das Ziel der Bachelorarbeit ist der Entwurf und Implementation eines Digitalen Zwilling Frameworks für unbemannte Luftfahrzeuge im MARS Framework.

Das Framework soll in der Lage sein ein physisches UAV in einem virtuellen Modell zu repräsentieren und durch einen kontinuierlichen Datenaustausch den aktuellen Zustand zu simulieren. Ein 3D Modell des physischen UAV soll entwickelt werden, damit die Bewegung überwacht und visualisiert werden kann. Des Weiteren soll das Framework eine Schnittstelle anbieten, um das Modell zu erweitern. Das Modell soll dynamisch angepasst und an den physischen UAV übertragen werden können. Für die direkte Interaktion des Digitalen Zwilling soll das System über eine Schnittstelle für Benutzereingaben verfügen.

Das fertige System soll zum Schluss ein Anwendungsfall demonstrieren. Es wird ein Jogging Trainer entwickelt, der einen Menschen beim Joggen begleitet. Durch die Analyse von fitnessbezogenen Daten des Benutzers soll das UAV ihr Flugverhalten anpassen. Der Benutzer soll durch das Flugverhatlen zum längeren Durchhalten animiert werden, 

Das fertige Modell soll in der Lage sein durch zukünftige Anforderungen und Anwendngen erweitert zu werden zu können. Zu den möglichen Erweiterungen gehören zum Beispiel das Hinzufügen neuer Datenquellen oder der Entwicklung weitere und komplexere Aktionen für das UAV.

% \input{chapters/0introduction/begriffe.tex}
\section{Abgrenzung}

% Möchte keine simulationsfähigen Digitalen Zwilling erstellen 
\input{chapters/0introduction/gliederungDerArbeit.tex}

\chapter{Begriffserklärung}


\section{Unbemanntes Luftfahrzeug}
Ein unbemanntes Luftfahrzeug, um englischen als Unmanned Aerial Vehicle oder in der Literatur als UAV abgekürzt, bezeichnet ein Luftfahrzeug, das autonom oder aus eine Distanz von einem Piloten gesteuert wird.

\section{Quadrokopter}
Unter der Quadrokopter sind unbemannte Luftfahzeuge, zum Fliegen von vier verticalgerichtete Rotoren getragen werden. Durch die Ausrichtung der Rotoren kann ein Quadrokopter sich  im dreidimensionalen Raum frei bewegen oder an einer beliebigen Position schweben. 

\subsection{funktionsweise}


\section{Digitaler Zwilling}

Das Konzept des Digitalen Zwilling, das im Jahr 2002 von Micheal Grieves formuliert wurde, beschreibt eine virtuelle Repräsentation eines nicht domänenspezifischen, physikalischen Objekts. Anders als ein Modell, das sein physikalisches Vorbild nur im Initialzustand gleicht, wandelt der Zustand des Digitale Zwilling sich mit seinem physikalischen Gegenstück und bleibt bis zu seinem Lebensende zusammen. 
Ein bidirektionaler Informationenaustausch ermöglicht es dem Digitalen Zwilling Einfluss am realen System zu nehmen. 

% !!! Zitat einer ähnlichen Definition 

\subsection{Human-Drone Interaction}



\input{chapters/MaterialUndMethoden/MaterialUndMethoden.tex}
\chapter{Anforderungsanalyse}

\section{Funktionale Anforderungen}

\mulcom {
\begin{table}[h]
\begin{tabular}{ l p{12cm}}
    ID & Begriff  \\
    \hline
    F01 & Der DZ fragt in periodischen Abständen den Status des UAV ab. \\
    F02 & Der DZ legt eine Log-Datei für die Zustandsänderungen des UAV ab. \\
    F04 & Trennt sich die Kommunikation zwischen dem UAV und dem DZ während des Fluges, wird der letzte Zustand des UAV gespeichert und versucht einen letzten Befehl zum laden zu senden. \\
    F05 & Das virtuelle UAV kann alle Operationen der physikalischen UAV ausführen. \\
    F06 & Falls der DZ seinen Zustand wechselt, wird die Aktion als Operation am PZ ebenso ausgeführt. \\
    F07 & Fall der PZ seinen zustand ändert, passt sich der Zustand des DZ an. \\
    F08 & Falls der DZ gestartet wird, verbindet sich das System mit dem physikalischen UAV \\
    F09 & Falls der DZ sich nach 5 Sekunden keine Verbindung zum PZ aufbauen kann, wird der Prozess beendet. \\
    F10 & Falls die Batterie der PZ bei 10Prozent liegt, wird der UAV automatisch gelandet u \\
    F11 & Die Statuswerte werden auf der Konsole angezeigt. \\
    F12 & Die Videoübertragung des PZ wird auf dem Monitor in Echtzeit übertragen. \\
    F13 & Der DZ wird visuell dargestellt. \\
    F14 & Verliert der UAV die Sicht zum Läufer, landet die Drohne aus Sicherheitsgründen. \\    
\end{tabular}
\caption{Funktionale Anforderungen}\label{table:Funktionale Anforderungen}

\end{table}
}
\newpage

\section{Nicht funktionale Anforderungen}

\begin{table}[h]
    \begin{tabular}{ l p{12cm}}
        ID & Beschreibung  \\
        \hline
        NF01 & Der Digitale Zwilling akzeptiert nur aktuelle Nachrichten und verwirft alle Nachrichten, die älter als 2 Sekunden sind. \\
        NF02 & Der Digitale Zwilling kann durch benutzerspezifische Anforderungen erweiter werden \\   
    \end{tabular}
    \caption{Nicht funktionale Anforderungen}\label{table:nicht Funktionale Anforderungen}
    
\end{table}

% \textbf{Erweiterbarkeit}

% \textbf{Zuverlässig}


% \section{Use Case}

% Test Szenarien

\chapter{Konzeption}

\section{Systemarchitektur}

\subsection{Quadroter}
Für den Entwicklung des Systems wird eine Tello Drohne von Rize verwendet. Die Drohne verfügt über einen Höhenmesssensor und eine Monokamera. 

% Warum wurde diese Drohne gewählt.

% Kommunikationswege

% Welche vor und Nachteile hat die Drohne?

\subsection{MARS Framework}

\subsection{Konzeptvorschlag des DTSUAV}

\subsubsection{Grundkonzept eines Digitalen Zwilling System Frameworks}

Ein bidirektionaler Informationenaustausch ermöglicht es dem Digitalen Zwilling Einfluss am realen System zu nehmen. 

% Beschreibe das Konzept des Digitalen Zwillings.


% !!! Zitat einer ähnlichen Definition  einfügen

%!!! nicht so ausführlich

\subsubsection{Konzeptvorschlag des DTSUAV}

%!!! must have
\input{chapters/5implementierung/implementierung.tex}
\chapter{Evaluation}

\section{Codeevaluation}

Der Quellcode wird mit Unit Testes gestest.

\chapter{Diskussion und Ausblick}

\input{chapters/DiskussionUndAusblick/zusammenfassung.tex}
\input{chapters/DiskussionUndAusblick/diskussion.tex}
\input{chapters/DiskussionUndAusblick/ausblick.tex}


% Add additional chapters here

\bibliographystyle{plain}
%\bibliographystyle{dinat}
\bibliography{literature}

% Appendix
\appendix
\input{appendix/example_appendix}

\IGlossary

\Istatement

\end{document}
