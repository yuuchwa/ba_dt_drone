\section{Motivation}

Unbemannte Luftfahrzeuge, im englischen Unmanned Aerial Vehicle, kurz UAV, werden vermehrt in verschiedenen Anwendungsfeldern eingesetzt. Besonders sogenannte Social Drones werden eingesetzt, um Menschen in ihrem Alltag zu unterstützen, indem sie beispielsweise für den Menschen Einkaufen gehen (zitat) oder Sachen Transportieren. Außerdem können Social Drones Aufgaben übernehmen, die Menschen mit einer körperlicher Beeinträchtung nicht mehr können. (Beispiel)

In Zukunft sollen Social Drones komplexere Aufgaben übernehmen können. In dieser Arbeit wird ein Digitaler Zwilling Framework für unbemannte Luftfahrzeuge vorgestellt, der dieses Problem angeht.

\mulcom{
Die Unmanned Aerial Vehicle, UAV oder zu deutsch unbemanntes Luftfahrzeuge, sind durch ihre ständige Forschung und Weiterentwicklung für den alltäglichen Gebrauch zugänglich geworden. Anfangs nur für den militärischen Einsatzzweck bestimmt,haben sich neue Anwendungsgebiete erschlossen. 


Zum Beispiel entwickelt das US amerikanische Unternehmen Amazon an Lieferungen per Drohnen \cite{2022AmaPriAirPreForDroDel}. Das Unternehmen erhofft sich damit Pakete sicherer und schneller liefern zu können. 

Erste konkrete Anwendungen wurden in den letzten Jahren umgesetzt, darunter Anwendungen, bei der die für Menschen Einkäufe erledigen sollt e(zitieren), Gegenstände herholen (zitieren) oder für Menschen mit einer Seheinschränkung Wege oder Gegenstände finden (zitieren). Obgleich in diesen Experimenten die Drohne noch von einem Piloten gesteuert werden, gibt es großes Interesse die Drohnen diese Aufgaben autonom erledigen können. 


Unteranderem entstanden Soziale Drohnen, die für die Unterstützung des Menschen entwickelt wurden. Erste konkrete Anwendungen wurden in den letzten Jahren umgesetzt, darunter Anwendungen, bei der die für Menschen Einkäufe erledigen sollt e(zitieren), Gegenstände herholen (zitieren) oder für Menschen mit einer Seheinschränkung Wege oder Gegenstände finden (zitieren). Obgleich in diesen Experimenten die Drohne noch von einem Piloten gesteuert werden, gibt es großes Interesse die Drohnen diese Aufgaben autonom erledigen können. 

(Überleitung zu Digitalen Zwillinge)
\\
(Einführung zu Digitalen Zwillinge)
\\
(Beispiel für die ersten Versuche von Digitalen Zwillingen mit UAV nennen)
\\

}


\mulcom{

Darin einbeschlossen können diese Drohnen nicht nur durch Algorithmen ihre Aufgaben immer besser ausführen, sondern auch die Preferenzen des Benutzers mit berücksichten. 

Ein Paradigma das sich mit der eigenen oder Überwachung anderer Objekte beschäftigt, ist der Digitale Zwilling.

Obwohl es erste gemeinsame Konzepte schon vor zehn Jahren gab, als die NASA den Digitalen Zwilling dazu verwenden wollte eine virtuelle Repräsentation von militärischen Luftfahrzeugen zu entwickeln \cite{Edw2012TheDigTwiParForFutNasAndUSAirForVeh}.

Durch die Zuwachs an UAV im öffentlichen Raum bildete sich daraus das neue Forschungsgebiet der Human-Drone Interaction. Die Human-Drone Interaktion untersucht die Interaktion zwischen UAV und Menschen und haben in der Vergangenheit \cite{Tezza2019StateOfTheArtHumanDroneInter}. Vorhergegangene Arbeiten haben ein positives Bild von UAV in der Öffentlichkeit gezeigt (Paper zitieren) und ein Interesse als Begleiter bei sportlichen Aktivtäten. Diese Arbeit setzt auf die Vorarbeit von 
}
