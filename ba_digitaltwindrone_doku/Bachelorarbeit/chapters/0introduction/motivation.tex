\section{Motivation}

Seit Jahren sind Unbemannte Luftfahrzeuge (eng. Unmanned Aearial vehicle, UAV), hier umgangsprachlich als Drohne bezeichnet, nicht mehr nur Speilzeuge der Regierung und dienen militärischen Einsatzzwecken, sodern findet durch die günstigen Bauteile auch Anreize bei vielen Hobbytechnikern, Technikintusiaten oder Menschen, die ein solches Spielzeug einfach mal benutzen wollen \cite{Gadiraju2021Understanding}. Dadurch werden Drohnen nicht mehr nur von Experten genutzt, sondern auch von Leihen, mit den Unterschliedlichsten Ansprüchen. Einige wollen Lernen mit einer Drohne umzugehen, andere nutzen die Drohne als Assisten im Alltag und wollen sie autonomisieren. Um diese Vielfalt an Anwendungszwecken zu bedienen, braucht man ein modulares Modell um auf einfacher Weise Anforderungen nach den Kriterien der Kunden anpassen zu können. Diese müssen auch in Echtzeit abgearbeitet werden. Darum eignet sich ein Digitaler Zwilling. Ein Digitaler Zwilling kann in Echtzeit Informationen mit der Drohne austauschen. Damit können der Drohne Regeln vorprogrammiert werden, die die Drohne in Echtzeit einhalten muss. 