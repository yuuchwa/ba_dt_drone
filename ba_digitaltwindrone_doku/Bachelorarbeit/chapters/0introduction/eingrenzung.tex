\section{Eingrenzung}

\mulcom {
Der Mittelpunkt der Bachelorarbeit liegt in der Konzeption und Entwicklung einer Architektur für den Social Companion, um Human Digital Twins einbinden zu können. Der Digitale Zwilling zur Leistungssteigung des Läufers soll ein anwendungsbezogene Beispiel sein, um einen möglichen Einsatzzweck für soziale Interaktion mit dem Menschen zu zeigen. Aus diesem Grund wird die tatsächliche Effektivität im Verhalten der Drohne zum Animieren zur Leistungserhöhung vernachlässigt.
}

Zu einem Digitalen Zwilling gehört, nach Grieves Idealvorstellung, eine virtuelle Umgebung hinzu, die die physikalische Umgebung des phyiskalischen Objekts nachahmt.
Für das Modell des Digitalen Zwilling ist die virtuelle Umgebung, notwendig, um die Parameter des virtuellen Objekts optimal zu modifizieren. 

Jedoch wird diese Komponente bewusst in dieser Bachelorarbeit außen vor gelassen, weil sie der Umfang der Arbeit überziehen würde.

Der Mittelpunkt der Arbeit ist die Entwicklung einer ersten Architektur, bei der einfache Informationenaustausch zwischen physikalischen und digitalen Zwilling möglich ist und eine Schnittstelle für die Entwicklung von autonomen Aktionen anbietet. 

% Möchte keine simulationsfähigen Digitalen Zwilling erstellen 