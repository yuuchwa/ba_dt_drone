\section{Ziel der Arbeit}

Für die Bachelorarbeit soll die Fitness eines Läufers als Digitalen Zwilling entwickelt werden. Getarnt als UAV begleitet der Digitale Zwilling den Läufer während seiner Laufeinheiten und sammelt seine Vitalwerte, um ihn durch ein bestimmtes Flugverhalten des UAV anzuspornen. Das System soll im MARS Framework konzipiert und entwickelt werden. 

Unterteilt wird die Arbeit in drei Teilbereiche:
Im ersten Teil soll die Basis des Digitalen Zwillings aufgebaut werden. Dazu gehört das Sammeln der benötigten Informationen und das Einrichten der Kommunikation und Ansteuerung des UAV.

Im zweiten Teil wird ein Digitale Zwilling konstruiert. Dieser Bilder verschiedene Kategorien für den Zustand des Läufers ab und modelliert verschiedene Wege, um den Benutzer als Begleiter zu unterstützen. 

Im letzten Teil erfolgt das Trainieren des Digitalen Zwillings.
