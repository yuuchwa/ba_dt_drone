\section{Zielsetzung}

\mulcom {

% Was soll gemacht werden?
Für die Bachelorarbeit so ein Prototype eines Human Digital twin entwickelt werden, der einen Benutzer beim Laufen begleiten und unterstützen soll.  Das System analysiert das Verhalten des Läufers.

% Wie macht es das
Der Digtiale Zwilling nutzt eine Drohne für die Interaktion mit dem physikalischen Zwilling, dem Menschen. Während des Laufens sammelt das System fitnessbezogene Informationen und modifiziert das Flugverhalten der Drohne, um den Läufer zum längeren Durchhalten zu animieren. Es wird darauf wertgelegt, Disziplienen der Human-Drone Interaction in das Verhalten einzusetzen. 

% Was ist das Ziel, was ich erwarte
Das Ziel der Arbeit liegt in der Entwicklung eines Social Companions, der einen Benutzer bei Laufaktivitäten begleitet. Das System soll nicht nur die Leistung des Benutzers verbessern, sondern durch ein positives Lauferlebnis durch soziale Interaktionen geben. 

}

Das Ziel der Bachelorarbeit ist der Entwurf und Implementation eines Digitalen Zwilling Frameworks für Unmanned Aerial Vehicles im MARS Framework.

Das Framework soll in der Lage sein ein physikalisches UAV in einem virtuellen Modell zu repräsentieren und durch einen kontinuierlichen Datenaustausch den aktuellen Zustand simulieren kann. Ein 3D Modell des physikalischen UAV soll entwickelt werden, damit die Bewegung überwacht und visualisiert werden kann. Des Weiteren soll das Framework eine Schnittstelle anbieten, um das Modell zu erweitern. Der Modell soll dynamisch anzupassen werden können und die optimierten Parameter an das physikalische UAV versendet werden können. Zur Interaktion und Überwachung des physikalischen Zwilling soll das System über eine Schnittstelle verfügen, um Human-Drohne Interaction Konzepte umzusetzen.

Das Framework soll insbesondere als Social Drohne genutzt werden, daher wird ein Anwedungsfall Entwickelt, der Konzepte aus der Human-Drone Interaction einfließen lässt.

Das Anwendungsbeispiel demontriert den Digitalen Zwilling als Jogging Trainer. Die Drohne soll einen Menschen beim Joggen begleiten und durch die Analyse von fitnessbezogenen Daten ein Flugverhalten einstellen, das den Benutzer zu längeren Durchhalten animiert.

Das fertige Modell soll in der Lage sein durch spätere Anforderungen und Anwendngen erweitert zu werden zu können. Zu den möglichen Erweiterungen gehören das Hinzufügen neuer Datenqzellen und der Entwicklung weitere und komplexere Aktionen für das UAV.

