\section{Ziel der Arbeit}

Für die Bachelorarbeit soll ein Digitaler Zwillings für ein Unbemanntes Luftfahrzeug (UAV) im MARS Framework konzeptioniert und entwickelt werden. Der Digitale Zwilling soll das Verhalten des UAV in einer digitale Simulation abbilden und ermöglichen das Fahrzeug durch weitere Funktionalitäten zu erweitern. Dabei soll Architektur des Digitalen Zwillings nicht domänenspezifisch sein. \newline
Die Entwicklung lässt sich in drei Aufgabenteile unterteilen.
Im ersten Teil wird ein Datenmodell des UAV entwickelt. Hierfür werden Sensordaten des physikalischen UAV gesammelt und im Datenmodell gespeichert. Diese Informationen werden ausgewertet und einem Zustand zugeordent. \newline
Der zweite Teil der Aufgabe besteht aus der Entwicklung einer digitalen Entität, die geometrisch und physikalischen den realen UAV darstellt. Hierbei wird eine Visualisierung erstellt, die das Verhalten des UAV animiert. \newline
Im letzten Teil soll ein Anwendungsbeispiel entwickelt werden. Das Beispiel soll das fertige System demonstrieren und dient der Systemevaluation.


%Die Arbeit lässt sich in drei Teilbereiche untergliedern.

% Die erste Teilaufgabe besteht aus dem Sammeln von Daten und der Erstellung eines Repräsentativen Zustandsgraphen. Dazu muss der Digitale Zwilling kontinuierlich die Zustandsinformationen des Quadrocopter abfragen und sie in das eigene Datenmodell abbilden.
% Ggf müssen aus den Informationen weiter Informationen berechnet werden. Mit hoher Wahrscheinlichkeit können die meisten Daten nur über die Kamerabilder ermittelt werden, weil die Drohne sonst kaum Informationen liefert

% Der dritten Teil wird die aquirierten Daten aus dem Datenmodell ausgewertet und der Zustand des physikalischen Objekts bestimmt. um Repräsentation der Zustände des physikalischen Objekts
