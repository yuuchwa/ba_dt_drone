\section{Zielsetzung}

\mulcom {

% Was soll gemacht werden?
Für die Bachelorarbeit so ein Prototype eines Human Digital twin entwickelt werden, der einen Benutzer beim Laufen begleiten und unterstützen soll.  Das System analysiert das Verhalten des Läufers.

% Wie macht es das
Der Digtiale Zwilling nutzt eine Drohne für die Interaktion mit dem physikalischen Zwilling, dem Menschen. Während des Laufens sammelt das System fitnessbezogene Informationen und modifiziert das Flugverhalten der Drohne, um den Läufer zum längeren Durchhalten zu animieren. Es wird darauf wertgelegt, Disziplienen der Human-Drone Interaction in das Verhalten einzusetzen. 

% Was ist das Ziel, was ich erwarte
Das Ziel der Arbeit liegt in der Entwicklung eines Social Companions, der einen Benutzer bei Laufaktivitäten begleitet. Das System soll nicht nur die Leistung des Benutzers verbessern, sondern durch ein positives Lauferlebnis durch soziale Interaktionen geben. 

}


Das Ziel der Bachelorarbeit ist der Entwurf eines Frameworks zur Integration Digitaler Zwillinge für unbemannte Luftfahrzeuge. 
Das Framework soll ins besondere die Integration von Human Digital Twins ermöglichen, um sogenannte Social Companions zu entwickeln.
Ein Social Companion nutzt Anforderungen der Human-Drone Interaction, um in der Interaktion mit einem Menschen angemessen und verständlich zu handeln.

Das fertige Modell soll in der Lage sein durch spätere Anforderungen und Anwendngen erweitert zu werden. Zu den möglichen Erweiterungen gehören neue Informationsquellen oder weitere und komplexere Aktionen des UAV.

Des Weiteren soll die Architektur durch ein Anwendungsbeispiel demonstriert werden. In dem Beispiel soll das unbemannte Luftfahrzeug durch Markierungen einen Menschen erkennen können und während des Gehenes verfolgen.