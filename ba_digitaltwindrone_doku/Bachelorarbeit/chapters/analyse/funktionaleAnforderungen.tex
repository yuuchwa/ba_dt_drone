\section{Funktionale Anforderungen}

\renewcommand{\arraystretch}{1.4}
\begin{table}[H]
\begin{tabular}{ p{0.5cm} p{3cm} p{11cm}}
    ID & Komponente & Beschreibung  \\
    \hline
    F01 & Digitaler Zwilling & Der DZ fragt in periodischen Abständen Statuswerte des physischen UAV. \\
    F02 & Digitaler Zwilling & Der DZ speichert alle erfassten Statuswerte des physischen UAV ab. \\
    F03 & Digitaler Zwilling & Der DZ seinen letzten Zustand wiederherstellen, wenn der Zustand mit dem des PZ übereinstimmt. \\
    F04 & Digitaler Zwilling & Endet oder bricht die Kommunikation zwischen dem PZ und dem DZ ab, wird der letzte Zustand des UAV gespeichert und der UAV landet zum automatischen Landen befohlen. \\
    F05 & Digitaler Zwilling & Fall der physischen UAV seinen Zustand ändert, passt sich der Zustand des DZ an. \\
    F06 & Phys. UAV & Falls der DZ seinen Zustand aktualisiert, wird die Änderung an den physischen UAV gesendet und ausgeführt. \\
    F07 & Phys. UAV & Falls die Batterie der PZ bei 10Prozent liegt, wird der UAV automatisch gelandet. \\
    F08 & Visualisierung & Das virtuelle UAV besitzt die gleichen kinetischen und physischen Eigenschaften wie die des physischen UAV. \\
    F09 & Visualisierung & Alle Bewegungen und Zustandsänderungen des physischen UAV werden vom virtuellen UAV gespiegelt. \\
    F10 & Visualisierung & Für die Visualisierung des physischen UAV im virtuellen Raum wird ein virtueller UAV erstellt. \\
    F11 & Kommunikation & Falls der DZ gestartet wird, verbindet sich der DZ mit dem physischen UAV. \\
    F12 & Kommunikation & Falls der DZ sich nach 5 Sekunden keine Verbindung zum PZ aufbauen kann, wird der Prozess beendet. \\
    F13 & Benutzerausgabe & Die Videoübertragung des physischen UAV wird in Echtzeit übertragen. \\
    F14 & Benutzerausgabe & Der Zustand und die Attributel des physischen UAV werden auf dem Rechner angezeigt.\\
    F15 & Fehlerzustand & Falls das physische UAV nach 5 gesenden Befehlen keine Antwortnachricht zuruckliefert, wird der physische zum Landen befohlen. \\
    F16 & Fehlerzustand & Falls der DZ einen unbekannten Fehlercode erhält, wird das physikalsche UAV zum Landen befohlen. \\
    F17 & Fehlerzustand & Falls sich der physische UAV in einem unbekannten Zustand befindet, wird sie zum Landen befohlen.
\end{tabular}
\caption{Funktionale Anforderungen}\label{table:Funktionale Anforderungen}
\end{table}

\subsection{Use Case Beschreibung}

\begin{usecase}
    \name{Digitalen Zwilling und UAV verbinden}
    \actor{UAV, Benutzer, System}
    \udescription{Starten und verbinden der Drohne mit dem  Digitalen Zwilling}
    \precondition{Das UAV ist eingeschaltet. Der Computer ist über WLan mit dem UAV verbunden}
    \scenario{
        \item System show something
        \item User do this
        \item System do that
    }
    \result{Digitaler Zwilling und UAV sind verbunden.}
    \extensions{
        \item[3a] If something do something
    }
    \exceptions{
        \item[2.1] System message: "Nope"
        \item[2.2] System message: "Bad action"
    }
\end{usecase}

\mulcom{
    \begin{usecase}
        \name{Titel}
        \actor{UAV, Benutzer, System}
        \udescription{Starten und verbinden der Drohne mit dem  Digitalen Zwilling}
        \precondition{Das UAV ist eingeschaltet. Der Computer ist über WLan mit dem UAV verbunden}
        \scenario{
            \item System show something
            \item User do this
            \item System do that
        }
        \result{Digitaler Zwilling und UAV sind verbunden.}
        \extensions{
            \item[3a] If something do something
        }
        \exceptions{
            \item[2.1] System message: "Nope"
            \item[2.2] System message: "Bad action"
        }
    \end{usecase}
}