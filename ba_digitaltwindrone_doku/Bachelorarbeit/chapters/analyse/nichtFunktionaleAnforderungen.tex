\section{Nicht funktionale Anforderungen}

Für die Güte einer Software sind nicht nur der funktionale Umfang zu berücksichtiegen, sondern auch die nicht-funktionale Anforderungen, die die Qualität einer Software bestimmt. Die Qualität können sich auf verschiedene Aspekte des Systems beziehen. Bei einem Digitalen Zwilling sind besonders viele nicht-funktionale Anforderungen wichtig, da sie durch viele Aufgaben viele Anforderungen berücksichtigen müssen. Im folgenden werden die wichtigsten nicht-funktionale Anforderungen für das System besprochen

**Performanz**

Ausführlich auf die Wichtigkeit der Performance eingehen, nicht nur in der Kommunikation, sondern auch in der  Berechnung und Auswertung.


- Der Digitale Zwilling muss ein kurzes Zeitverhalten in der Entscheidungsfindung und der Steuerung habe. Die Drohne fliegt frei in der Luft und ist dadurch von Hindernissen und dem Sturz gefährdet. Daher müssen Änderungen der Tello Drone in kurzer Zeit verarbeitet werden können und präzise im System gespiegelt werden können.
    - [https://guidingcode.com/non-functional-requirements-in-software-engineering/](https://guidingcode.com/non-functional-requirements-in-software-engineering/)
    - [A review on drones controlled in real-time](https://www.notion.so/A-review-on-drones-controlled-in-real-time-5fb8333612e04e959baafc8b895cc4a1) (noch nicht gelesen)
- Zeitperformant, sodass Anforderungen schnell abgearbeitet werden und alte Nachrichten verworfen werden.

**Verfügbarkeit/Zuverlässigkeit**

Die Verfügbarkeit beschreibt die Persistenz gegen Systemausfällen, sowie dessen Erreichbarkeit während eines Problem. Bei Zwischenausfällen ist das System nicht mehr Antwortfähig und kann im schlimmsten Fall einen Unfall verursachen. 

In einem Digitalen Zwilling bedeutet Verfügbarkeit eine ständige Bereitschaft von wichtigen Modulen und eine ständige Kommunikationsverbindung zur Tello-Drohne. 

Ersteres Bedeutet, dass die Komponenten zur Aktionsfindung und Auswertung des Zustandes ständig abrufbar sein müssen.

Die Kommunikation zur Tello-Drohne ist ebenfalls ein wichtiger Faktor, denn einAusfall der kommunikation kann zu einem Unfall an der Tello-Drohne führen.

**Funktionalität**

**Modularer Aufbau**

**Usability**

Ein menschlicher Pilot soll die Drohne ständig überwachen und eingreifen können, falls eine unverhersehene Situation eintritt. Dazu muss die Steuerung intuitiv sein, damit der Benutzer plötzliche Eingriffe schnell umsetzen kann und bei den Eingaben nur wenige Fehler macht. Ebenso braucht der Nutzer eine übersichtliche und einfach verständliche Überwachungsoberfläche, um den Zustand der Drohne einfach ablesen zu können.

**Gute Analyse-Funktion**



% \textbf{Erweiterbarkeit}

% \textbf{Zuverlässig}

