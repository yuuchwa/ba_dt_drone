\section{Zielsetzung}

Das Ziel der Bachelorarbeit ist der Entwurf und Implementation eines Digitalen Zwilling Frameworks für unbemannte Luftfahrzeuge im MARS Framework.

Das Framework soll in der Lage sein ein physisches UAV in einem virtuellen Modell zu repräsentieren und durch einen kontinuierlichen Datenaustausch den aktuellen Zustand zu simulieren. Ein 3D Modell des physischen UAV soll entwickelt werden, damit die Bewegung überwacht und visualisiert werden kann. Des Weiteren soll das Framework eine Schnittstelle anbieten, um das Modell zu erweitern. Das Modell soll dynamisch angepasst und an den physischen UAV übertragen werden können. Für die direkte Interaktion des Digitalen Zwilling soll das System über eine Schnittstelle für Benutzereingaben verfügen.

Das fertige System soll zum Schluss ein Anwendungsfall demonstrieren. Es wird ein Jogging Trainer entwickelt, der einen Menschen beim Joggen begleitet. Durch die Analyse von fitnessbezogenen Daten des Benutzers soll das UAV ihr Flugverhalten anpassen. Der Benutzer soll durch das Flugverhatlen zum längeren Durchhalten animiert werden, 

Das fertige Modell soll in der Lage sein durch zukünftige Anforderungen und Anwendngen erweitert zu werden zu können. Zu den möglichen Erweiterungen gehören zum Beispiel das Hinzufügen neuer Datenquellen oder der Entwicklung weitere und komplexere Aktionen für das UAV.
