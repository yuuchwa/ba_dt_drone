\section{Systemevaluation/Case Study}

(Beschreibung des Versuchs)

\mulcom {
\begin{table}[h]
    \begin{tabular}{ l p{12cm}}
        ID & Begriff  \\
        \hline
        F01 & Der DZ fragt in periodischen Abständen den Status des UAV und des Menschen ab. \\
        F02 & Der DZ speichert den Status des PZ ab. \\
        F03 & Der DZ seinen letzten Zustand wiederherstellen, wenn der Zustand mit dem des PZ übereinstimmt. \\
        F04 & Endet oder bricht die Kommunikation zwischen dem PZ und dem DZ ab, wird der letzte Zustand des UAV gespeichert und der UAV landet aus automatisch. \\
        F05 & Der DZ alle Operationen des PZ ausführen. \\
        F06 & Falls der DZ seinen Zustand wechselt, wird die Aktion als Operation am PZ ebenso ausgeführt. \\

        F08 & Falls der DZ gestartet wird, verbindet sich das System mit dem physischen UAV \\
        F09 & Falls der DZ sich nach 5 Sekunden keine Verbindung zum PZ aufbauen kann, wird der Prozess beendet. \\
        F10 & Falls die Batterie der PZ bei 10Prozent liegt, wird der UAV automatisch gelandet u \\
        F11 & Die Statuswerte werden auf der Konsole angezeigt. \\
        F12 & Die Videoübertragung des PZ wird auf dem Monitor in Echtzeit übertragen. \\
        F13 & Der physische UAV wird als virtuelle UAV dargestellt. \\
        F00 & Das virtuelle UAV besitzt die gleichen kinetischen und physischen Eigenschaften wie vom physischen UAV. \\
        F07 & Fall der PZ seinen zustand ändert, passt sich der Zustand der virtuellen UAV an. \\ 
    \end{tabular}
    \caption{Funktionale Anforderungen für den Jogging Trainer}\label{table:Funktionale Anforderungen für den Jogging Trainer}
    
    \end{table}
}