\chapter{Verwandte Arbeit}

% Dieses Kapitel ist nicht mit dem Grundlagenkapitel (siehe n ̈achster Punkt) zu verwechseln. Es geht hier also nicht darum, Grundlagen des Arbeitsgebietes oder Grundlagen, die man f ̈ur die eigene L ̈osung ben ̈otigt, darzustellen, sondern in erster Linie darum, Arbeiten/Produkte/L ̈osungen vorzustellen, die ein  ̈ahnliches (oder im Extremfall sogar identisches) Thema haben wie die eigene Abschlussarbei

% Die Forschung zur Umsetzung eines Digitalen Zwillings steht noch weit am Anfang und es gibt nicht viele Beispiele. bishergige Arbeiten beziehen sich auf Architekturen nach der allgemeinen Vorgabe des Digitalen Zwillings. Also einer Archtektur bestehen aus virtueller, physischer Entität und dazu einen Layer in der benutzerspezifische Anforderungen umgesetzt werden können. 

% besonders die agenten basierte Modellierung für Produkt digitalen Zwilling sind wenig erkundet und es gibt kaum Beispiele für die Umsetzung eines solchen Systems
