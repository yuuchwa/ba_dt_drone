\section{Systemarchitektur}

\subsection{Quadroter}
Für den Entwicklung des Systems wird eine Tello Drohne von Rize verwendet. Die Drohne verfügt über einen Höhenmesssensor und eine Monokamera. 

% Warum wurde diese Drohne gewählt.

% Kommunikationswege

% Welche vor und Nachteile hat die Drohne?

\subsection{MARS Framework}

\subsection{Konzeptvorschlag des UAV}

\subsubsection{Grundkonzept eines Digitalen Zwilling Frameworks}

Zu den Komponenten eines Digitalen Zwilling Frameworks gehören ein realer Raum, ein virtueller Raum und einem Kommunikationsschnittstelle, die die beiden Räume verbindet \cite{Gri2016OriOfTheDigTwiCon}. Der Digitale Zwilling setzt sich aus mehreren virtuellen Räumen zusammen, die das Verhalten und die Eigenschaften des physischen Objekts im physischen Raum wiederspiegelt.
Verbunden durch eine bidirektionale Kommunikationsschnittstelle können sich beide Systemen gegenseitig austauschen und ihren aktuellen Zustand synchronisieren. Über Sensoren nimmt physischen Objekt seine Umgebung und sich selbst wahr und übermittelt diese Attribute an den Digitalen Zwilling, der seinen Zustand ändert. Wiederum kann der Digitale Zwilling im virtuellen Raum Prozesse optimieren und sie an das physische Objekt zurückschicken.
Basierend auf dieses Prinzip soll das vorgestellte Framework ein Modell für UAV besitzen und modelliert im virtuellen Raum die Eigenschaften des physischen UAV nach. Der UAV übermittelt kontinuierlich seine Zustandsinformationen an den Digitalen Zwilling.
Im Virtuellen Raum können Funktionalitäten entwickelt werden, die die Parameter optimieren können. Die modifizierten Parameter können anschließen an das physische Objekt zurückgesendet werden. Eine Visualisation bietet dem Benutzer die Bewegungen des physischen UAV zu verfolgen und seine Aktivitäten zu überwachen.

% Beschreibe das Konzept des Digitalen Zwillings.


% !!! Zitat einer ähnlichen Definition  einfügen

%!!! nicht so ausführlich

\subsubsection{Konzeptvorschlag des DTUAV}

%!!! must have