
\section{Unbemanntes Luftfahrzeug}
Ein unbemanntes Luftfahrzeug, im englischen als Unmanned Aerial Vehicle, sind Luftfahrzeuge, die von Piloten ferngesteuert werden oder autonom sind. \mulcom{Umgangssprachlich werden Umbemannte Luftfahrzeuge auch als Drohne bezeichnet.}

\section{Multicopter}
Unter der Multicopter werden unbemannte Luftfahzeuge bezeichnet, die im Flug von mindestens zwei vertikalgerichteten Rotoren getragen werden. Wegen der Ausrichtung der Rotoren können Multicopter im drei dimensonalen Raum in alle Richtungen fliegen. Der Vertikalflug ermöglicht dem Multicopter an einer festen Position zu bleiben und zu schweben.

% \subsection{funktionsweise}

\section{Human-Drone Interaction}
Die Human-Drone Interaction ist ein Forschungsfeld der Human-Robotic Interaction \cite{Tezza2019TheStaOfArtHumDro}
und beschäftigt sich mit der Erforschung und Evaluation neuer Interaktionswege mit einem UAV. Als Drohne sind in Human-Drone Interaction meistens Multicopter gemeint.

\subsection{Social Drone}
mulcom{
Social Drones. Die Absicht dieser Forschungsrichtung ist der Einsatz von UAV zur Unterstützung von Menschen im Alltagsleben. \cite{Ghafu2021SocCom}.}

\subsection{Digitaler Zwilling}

Das Konzept des Digitalen Zwilling wurde 2002 von Micheal Grieves eingeführt und beschreibt die Modellierung eines physikalischen, nicht domainspezifischen Objekts in einenm virtuellen Raum.  

Die virtuelle Repräsentation des physikalischen Objekts ermöglicht die Simulation unterschiedlicher Szenarien, um die bestmögliche Konfiguration des Systems für den gegebenen Zustand vorzunehmen und sie auf das physikalischen Objekt anzuwenden. 

\textbf{Physikalischer Raum} \\

\textbf{Virtuelle Raum} \\

\textbf{Modell des Digitalen Zwillings} \\

\textbf{Physikalischer Umgebung} \\

\textbf{Virtuelle Umgebung} \\


% welche technologischen Vorteile erhoffte sich Grieves?


