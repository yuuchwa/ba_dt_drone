
\section{Unbemanntes Luftfahrzeug}
Ein unbemanntes Luftfahrzeug, um englischen als Unmanned Aerial Vehicle oder in der Literatur als UAV abgekürzt, bezeichnet ein Fluggeräte, die autonom fliegen können oder die aus der Ferne von einem Piloten gesteuert werden. Umgangssprachlich werden Umbemannte Luftfahrzeuge auch als Drohne bezeichnet.

\section{Multicopter}
Unter der Multicopter werden im wesentlichen unbemannte Luftfahzeuge bezeichnet, die im Flug von mindestens zwei vertikalgerichtete Rotoren getragen werden. Wegen der Ausrichtung der Rotoren können Multicopter sich zusätzlich zu einer horizontalen Richtung auch vertikal bewegen. Der vertikale Flug erlaubt es dem Multicopter an einer beliebigen Position zu schweben. 

% \subsection{funktionsweise}

\section{Human-Drone Interaction}
Die Human Drone interaction ist ein Forschungsfeld in der  Human-Robotic Interaction und hat sich als selbstständiges Fachgebiet herausgebildet, da die Charakteristiken in der Interaktion mit einem UAV sich zu einen statisch, stationäre Roboter unterscheiden \cite{Tezza2019TheStaOfArtHumDro}. Als UAV sind vorangig Multicopter gemeint.
Die Human-Drone Interaktion erforscht ein breites Themengebiet um Verständnis und neue Schnittstellen für die Menschen und UAV zu entwickeln .

\subsection{Social Drone}
Im Forschungsfeld der Human-Drone Interaction beschreibt man einen Social Drohne ein auf die soziale Interaktion ausgelegtes UAV System. Die Absicht dieser Forschungsrichtung ist der Einsatz von UAV zur Unterstützung von Menschen im Alltagsleben. \cite{Ghafu2021SocCom}.

\subsection{Digitaler Zwilling}

Das Konzept des Digitalen Zwilling wurde 2002 von Micheal Grieves eingeführt und beschreibt die virtuelle Repräsentation eines nicht domänenspezifischen, physikalischen Objekts.

\textbf{Physikalische Entität}
\textbf{Digitale Entität}
\textbf{Modell des Digitalen Zwillings}

\textbf{Physikalischer Umgebung}
\textbf{Virtuelle Umgebung}


% welche technologischen Vorteile erhoffte sich Grieves?


