
\section{Unbemanntes Luftfahrzeug}
Ein unbemanntes Luftfahrzeug, um englischen als Unmanned Aerial Vehicle oder in der Literatur als UAV abgekürzt, bezeichnet ein Luftfahrzeug, das autonom oder aus eine Distanz von einem Piloten gesteuert wird.

\section{Quadrokopter}
Unter der Quadrokopter sind unbemannte Luftfahzeuge, zum Fliegen von vier verticalgerichtete Rotoren getragen werden. Durch die Ausrichtung der Rotoren kann ein Quadrokopter sich  im dreidimensionalen Raum frei bewegen oder an einer beliebigen Position schweben. 

\subsection{funktionsweise}


\section{Digitaler Zwilling}

Das Konzept des Digitalen Zwilling, das im Jahr 2002 von Micheal Grieves formuliert wurde, beschreibt eine virtuelle Repräsentation eines nicht domänenspezifischen, physikalischen Objekts. Anders als ein Modell, das sein physikalisches Vorbild nur im Initialzustand gleicht, wandelt der Zustand des Digitale Zwilling sich mit seinem physikalischen Gegenstück und bleibt bis zu seinem Lebensende zusammen. 
Ein bidirektionaler Informationenaustausch ermöglicht es dem Digitalen Zwilling Einfluss am realen System zu nehmen. 

% !!! Zitat einer ähnlichen Definition 

\subsection{Human-Drone Interaction}
