
Dieses Kapitel stellt in relevanten Grundlagen der Bachelorarbeit vor und dienen dem verständnis der nachfolgenden Kapitel. 

\section{Unbemanntes Luftfahrzeug}
Ein unbemanntes Luftfahrzeug, im englischen als Unmanned Aerial Vehicle, sind Luftfahrzeuge, die von Piloten ferngesteuert werden oder autonom sind.

\section{Quadrokopter}
Unter Quadrokoptern werden unbemannte Luftfahzeuge bezeichnet, die im Flug von vier vertikalgerichteten Rotoren getragen werden. Wegen der Ausrichtung der Rotoren können Multicopter sich im drei dimensonalen Raum frei bewegen. Der Vertikalflug ermöglicht dem Multicopter an einer festen Position zu verbleiben.

\subsection{Tello Drone}

\subsection{Bauteile}

Sensoren: Accelerometer, Gyroskop, Inertia Measurement uni, Magnetometer, Barometer,

Framework

Motoren und Propeller

Batterie

\subsection{funktionsweise}

Bewegungsrichtung: yaw, thrift, roll

\section{Human-Drone Interaction}
Die Human-Drone Interaction ist ein Forschungsfeld der Human-Robotic Interaction \cite{Tezza2019TheStaOfArtHumDro}
und beschäftigt sich mit der Erforschung und Evaluation neuer Interaktionswege mit einem UAV. Als Drohne sind in Human-Drone Interaction meistens Multicopter gemeint.

\section{Digitaler Zwilling}

    Das Konzept des Digitalen Zwilling wurde 2002 von Micheal Grieves eingeführt und beschreibt die Modellierung eines physischen, nicht domainspezifischen Objekts in einenm virtuellen Raum.  

    Die virtuelle Repräsentation des physischen Objekts ermöglicht die Simulation unterschiedlicher Szenarien, um die bestmögliche Konfiguration des Systems für den gegebenen Zustand vorzunehmen und sie auf das physischen Objekt anzuwenden. 

    \textbf{physischer Raum} \\

    \textbf{Virtuelle Raum} \\

    \textbf{Modell des Digitalen Zwillings} \\

    \textbf{physischer Umgebung} \\

    \textbf{Virtuelle Umgebung} \\

    \subsection{Definition und Geschichte}

    \subsection{Anwendungen}

    \subsection{Digitaler Zwilling für Unbemannte Luftfahrzeuge}

\section{Multi-Agent System}

\section{MARS Framgework}

    \subsection{Multi-Agend System with Mars}

    \subsubsection{Agent}

    \subsubsection{layer}

% welche technologischen Vorteile erhoffte sich Grieves?


