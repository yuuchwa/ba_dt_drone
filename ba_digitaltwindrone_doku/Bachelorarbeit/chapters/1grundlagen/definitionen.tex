
Dieses Kapitel stellt in relevanten Grundlagen der Bachelorarbeit vor und dienen dem verständnis der nachfolgenden Kapitel. 

\section{Digitaler Zwilling}

    Das Konzept des Digitalen Zwilling wurde 2002 von Micheal Grieves eingeführt und beschreibt die Modellierung eines physischen, nicht domainspezifischen Objekts in einenm virtuellen Raum.  

    Die virtuelle Repräsentation des physischen Objekts ermöglicht die Simulation unterschiedlicher Szenarien, um die bestmögliche Konfiguration des Systems für den gegebenen Zustand vorzunehmen und sie auf das physischen Objekt anzuwenden. 

    \textbf{physischer Raum} \\

    \textbf{Virtuelle Raum} \\

    \textbf{Modell des Digitalen Zwillings} \\

    \textbf{physischer Umgebung} \\

    \textbf{Virtuelle Umgebung} \\

    \subsection{Definition und Geschichte}

    \subsection{Anwendungen}

    \subsection{Digitaler Zwilling für Unbemannte Luftfahrzeuge}

\section{Multi-Agent System}

\section{MARS Framgework}

    \subsection{Multi-Agend System with Mars}

    \subsubsection{Agent}

    \subsubsection{layer}

% welche technologischen Vorteile erhoffte sich Grieves?


